\documentclass[11pt]{article}
\usepackage{graphicx}
\usepackage{parskip}

\usepackage{xeCJK} 
\setCJKmainfont{Noto Sans CJK TC}

\usepackage{geometry}
\geometry{
 a4paper,
% total={210mm,297mm},
 left=1in,
 right=1in,
 top=1in,
 bottom=1in
}

\usepackage{fancyhdr}
\chead{\raisebox{0.7mm}{Test Problemset}}
\pagestyle{fancy}

\usepackage{csquotes}

\begin{document}

\pagenumbering{gobble}
\begin{center}
    {\LARGE Problem B}\\
    {\Large Longest Monotonically Increasing Subsequence}\\
    {Time limit: 1 second}\\
    {Memory limit: 512 megabytes}
\end{center}

\textbf{\large Problem Description}

Let $a_1,\dots,a_n$ be a sequence of integers.
$a_{i_1},\dots,a_{i_k}$ is a monotonically increaing subsequence of 
$a_1,\dots,a_n$ if the following conditions are satisfied.
\begin{itemize}
\item $0<i_1<i_2<\cdots<i_k\le n$.
\item $a_{i_1}<a_{i_2}<\cdots<a_{i_k}$
\end{itemize}
Your task is to find the longest monotonically increasing subsequence of
a given sequence $a_1,\dots,a_n$. In other words, your program should output 
a monotontically increasing subsequence $a_{i_1},\dots,a_{i_k}$ such that $k$ 
is maximized. If there are multiple candidates, you may output any one of them.
For example, let us assume the given sequence is 
\verb+2,1,3,4,5+. You should output either \verb+2,3,4,5+ or \verb+1,3,4,5+,
since they are the only two longest monotonically increasing subsequences
of \verb+2,1,3,4,5+.

\textbf{\large Input Format}

The input is terminated by end-of-file, and there are at most 30 test cases.
Each test case consists of two lines. The first line contains exactly one 
positive integer $n$ indicating the length of the given sequence. The second 
line contains $n$ integers $a_1,\dots,a_n$ saparated by blanks.
You may assume that $n \le 24$ and $a_1,\dots,a_n\in\{0,\dots,99\}$.

\textbf{\large Output Format}

For each test case, output any longest monotonically increasing subsequence
of $a_1,\dots,a_n$. You should saparate the numbers by blanks.

\textbf{\large Sample Input}

\begin{verbatim}
5
2 1 3 4 5
5
2 1 3 4 5
5
1 2 3 4 5
5
5 4 3 2 1
\end{verbatim}

\textbf{\large Sample Output}
\begin{verbatim}
1 3 4 5
2 3 4 5
1 2 3 4 5
2
\end{verbatim}

\end{document}
